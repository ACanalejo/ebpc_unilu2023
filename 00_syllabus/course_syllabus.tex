% Options for packages loaded elsewhere
\PassOptionsToPackage{unicode}{hyperref}
\PassOptionsToPackage{hyphens}{url}
\PassOptionsToPackage{dvipsnames,svgnames,x11names}{xcolor}
%
\documentclass[
  12pt,
]{article}
\title{Electoral Behavior and Party Competition: Continuity and Change
in Western Europe\footnote{This version of the syllabus: 16 Feb 2023}\\
\vspace{15truemm}}
\usepackage{etoolbox}
\makeatletter
\providecommand{\subtitle}[1]{% add subtitle to \maketitle
  \apptocmd{\@title}{\par {\large #1 \par}}{}{}
}
\makeatother
\subtitle{Masterseminar Spring 2023

Wednesdays 14:15-16:00

Classroom HS12\footnote{Sessions 5 and 6 will take place in room 3.A05
  on Wednesday, 22nd March, from 13:30 to 17:00.}}
\author{}
\date{\vspace{-2.5em}}

\usepackage{amsmath,amssymb}
\usepackage{lmodern}
\usepackage{iftex}
\ifPDFTeX
  \usepackage[T1]{fontenc}
  \usepackage[utf8]{inputenc}
  \usepackage{textcomp} % provide euro and other symbols
\else % if luatex or xetex
  \usepackage{unicode-math}
  \defaultfontfeatures{Scale=MatchLowercase}
  \defaultfontfeatures[\rmfamily]{Ligatures=TeX,Scale=1}
\fi
% Use upquote if available, for straight quotes in verbatim environments
\IfFileExists{upquote.sty}{\usepackage{upquote}}{}
\IfFileExists{microtype.sty}{% use microtype if available
  \usepackage[]{microtype}
  \UseMicrotypeSet[protrusion]{basicmath} % disable protrusion for tt fonts
}{}
\makeatletter
\@ifundefined{KOMAClassName}{% if non-KOMA class
  \IfFileExists{parskip.sty}{%
    \usepackage{parskip}
  }{% else
    \setlength{\parindent}{0pt}
    \setlength{\parskip}{6pt plus 2pt minus 1pt}}
}{% if KOMA class
  \KOMAoptions{parskip=half}}
\makeatother
\usepackage{xcolor}
\IfFileExists{xurl.sty}{\usepackage{xurl}}{} % add URL line breaks if available
\IfFileExists{bookmark.sty}{\usepackage{bookmark}}{\usepackage{hyperref}}
\hypersetup{
  colorlinks=true,
  linkcolor={blue},
  filecolor={Maroon},
  citecolor={Blue},
  urlcolor={Blue},
  pdfcreator={LaTeX via pandoc}}
\urlstyle{same} % disable monospaced font for URLs
\usepackage[margin=1in]{geometry}
\usepackage{graphicx}
\makeatletter
\def\maxwidth{\ifdim\Gin@nat@width>\linewidth\linewidth\else\Gin@nat@width\fi}
\def\maxheight{\ifdim\Gin@nat@height>\textheight\textheight\else\Gin@nat@height\fi}
\makeatother
% Scale images if necessary, so that they will not overflow the page
% margins by default, and it is still possible to overwrite the defaults
% using explicit options in \includegraphics[width, height, ...]{}
\setkeys{Gin}{width=\maxwidth,height=\maxheight,keepaspectratio}
% Set default figure placement to htbp
\makeatletter
\def\fps@figure{htbp}
\makeatother
\setlength{\emergencystretch}{3em} % prevent overfull lines
\providecommand{\tightlist}{%
  \setlength{\itemsep}{0pt}\setlength{\parskip}{0pt}}
\setcounter{secnumdepth}{-\maxdimen} % remove section numbering
\usepackage{setspace}\onehalfspacing
\usepackage[utf8]{inputenc}
\usepackage{dcolumn}
\usepackage{booktabs}
\usepackage{longtable}
\usepackage{array}
\usepackage{multirow}
\usepackage{wrapfig}
\usepackage{float}
\usepackage{colortbl}
\usepackage{pdflscape}
\usepackage{tabu}
\usepackage{threeparttable}
\usepackage{threeparttablex}
\usepackage[normalem]{ulem}
\usepackage{makecell}
\usepackage{xcolor}
\usepackage{fancyhdr}
\pagestyle{fancy}
\fancypagestyle{plain}{\pagestyle{fancy}}
\fancyhead[LE,RO]{Álvaro Canalejo-Molero, MA \linebreak \href{mailto:alvaro.canalejo@unilu.ch}{\tt alvaro.canalejo@unilu.ch}}
\fancyhead[RE,LO]{University of Lucerne}
\usepackage{flafter}
\ifLuaTeX
  \usepackage{selnolig}  % disable illegal ligatures
\fi

\begin{document}
\maketitle

\hypertarget{course-description}{%
\section{Course description}\label{course-description}}

Elections in Europe have become increasingly unpredictable in the last
decades. In Germany, for example, the government negotiations escalated
from one month in 2013 to almost six months after the entry of the
radical party AfD in the \emph{Bundestag} in 2017. In other European
countries, such as Spain or the Netherlands, the number of parliamentary
parties has doubled in the last twenty years, while the Italian party
system has arguably collapsed twice in the same period. How has this
situation come about? What are its consequences? These changes are not
random but driven by societal transformations and the strategies of
political actors, such as social movements and new challenger parties.
In addition, they have not affected all of Europe equally but depend on
factors that vary across countries and over time. This seminar will
review the different `transformation waves' that have shaped Western
European party systems from the 1960s until now. It will then try to
make sense of these changes by exploring the most relevant (electoral)
demand-side and supply-side explanations discussed in the literature.
The goal is to provide students with the conceptual and empirical tools
to analyze the behaviour of voters and parties by following the
evolution of Western European party systems from a comparative
perspective.

\hypertarget{course-organization}{%
\section{Course organization}\label{course-organization}}

The course consists of a weekly seminar that will run during the Spring
term of 2023 (February 22nd - May 31st). The substantive focus of the
seminars is organized in four blocks. The first block will introduce
students to the main concepts in the study of parties and party systems
and review the main theories of party system formation. The second block
will review the different transformational waves that have affected
Western European party systems from the 1960s until now. The third block
will delve into the literature on political behaviour and party
competition to focus on the mechanisms behind party system stability and
change. The final block will deal with the measurement of party system
change and its normative implications.

\hypertarget{learning-outcomes}{%
\section{Learning outcomes}\label{learning-outcomes}}

By the end of this course, students will be able to:

\begin{itemize}
\item
  \textbf{(Knowledge)} Identify the main theoretical approaches to the
  study of party system formation and change, integrate supply and
  demand-side explanations for the study of electoral outcomes, and use
  the most relevant indicators to assess electoral change.
\item
  \textbf{(Competence)} Critically reflect on the different theories and
  methods reviewed during the course to assess the predictive validity
  of different approaches in time and space.
\item
  \textbf{(Research skills)} Read scientific articles with a critical
  perspective, develop their own hypotheses and connect them with the
  relevant literature, and design feasible ways to test them.
\item
  \textbf{(Communication)} Communicate complex concepts effectively to a
  broad audience, alone or in collaboration with colleagues.
\end{itemize}

\hypertarget{teaching-policy}{%
\section{Teaching policy}\label{teaching-policy}}

This course is designed as a seminar, not a lecture. It aims to foster
and refine critical thinking by providing a collaborative learning
environment. The lecturer will give a broad overview of the subject
matter in each lesson to guide the discussion, which the students will
drive. Therefore, students are expected to study the mandatory readings
before class. Slides are not intended to replace given texts and will
only be distributed after each session.

\hypertarget{evaluation}{%
\section{Evaluation}\label{evaluation}}

\hypertarget{mandatory-requirements-4-credits}{%
\subsection{Mandatory requirements (4
credits)}\label{mandatory-requirements-4-credits}}

To receive the credits, students are expected to fulfil the following
criteria:

\hypertarget{attend-all-the-sessions.}{%
\subsubsection{1. Attend all the
sessions.}\label{attend-all-the-sessions.}}

Attendance is mandatory. Students can miss a maximum of two sessions.
Missing more sessions without a justified certificate of absence implies
failing the course.

\hypertarget{study-the-mandatory-readings-before-each-session.}{%
\subsubsection{2. Study the mandatory readings before each
session.}\label{study-the-mandatory-readings-before-each-session.}}

\emph{Studying = reading + critical thinking}. Students can find the
texts in the materials folder in OLAT.

\hypertarget{write-two-response-papers.}{%
\subsubsection{3. Write two response
papers.}\label{write-two-response-papers.}}

Students must select two sessions and write a short response paper
(500-700 words) to the mandatory readings of the session.

A good response paper should:

\begin{itemize}
\item
  \emph{Start with a short summary of the paper}. It should not be a
  summary of the content but of the main thesis and findings of the
  paper.
\item
  \emph{Identify the contribution.} It should state clearly what bigger
  problem or question the author is contributing to addressing.
\item
  \emph{Evaluate the argument from a critical perspective.} I should
  identify the main contributions and limitations of the paper and
  proposes ways in which these limitations should be addressed either by
  the author or in future research.
\end{itemize}

Some of the readings may be less suited to this structure, for example,
book chapters or literature reviews. In these cases, the students are
encouraged to develop different critical evaluations of the text, such
as developing some of the open questions or hypotheses laid out in the
text.

The suggested readings are not mandatory. However, the students will
benefit from linking them to the main text in their response papers.

The \textbf{submission rules} are:

\begin{itemize}
\item
  Students must upload their response papers to the \emph{Response
  papers} folder in OLAT.
\item
  The deadline for submitting the response paper is the \textbf{Sunday
  before the corresponding session at noon (12:00)}.
\item
  Students cannot send papers for \emph{session 01} and \emph{session
  14}, which are free-readings sessions. See
  \protect\hyperlink{course-schedule}{Course schedule}.
\item
  One of the response papers must be submitted for a session before the
  \textbf{spring break (10-16th April)} and one afterwards.
\item
  The response papers cannot be sent for the same session as the
  student's presentation.
\item
  The file must be in Word. The title should include the session's
  number followed by the student's surname in capital letters and the
  response paper number (either 1 or 2). For example:
  \emph{session04\_CANALEJO\_1}
\end{itemize}

\hypertarget{conduct-one-presentation.}{%
\subsubsection{4. Conduct one
presentation.}\label{conduct-one-presentation.}}

Presentations can be done alone or in groups of two or three students,
depending on the number of students taking the course. The schedule will
be decided upon the {[}Introductory session{]}.

Presenters must critically re-examine the mandatory readings of the
session to locate a troubling or unsatisfactory element, then propose a
solution including a theoretical motivation and a feasible research
design. A discussion of the proposal will follow the presentation.
Presenters must upload the slides of the presentation \textbf{at least
24 hours before the presentation} to the \emph{Presentations} folder in
OLAT to allow the discussant to prepare.

For more information, see the section
\protect\hyperlink{guidelines-for-the-presentation}{Guidelines for the
presentation}. The use of this presentation to develop an independent
research project is encouraged (e.g.~future seminar papers or thesis).

\emph{Sessions 01, 02 and 14} are free from presentations.

\hypertarget{serve-as-discussant-of-one-presentation.}{%
\subsubsection{5. Serve as discussant of one
presentation.}\label{serve-as-discussant-of-one-presentation.}}

Each presentation will be assigned a discussant. The discussant must
study the research proposal of the presenters in advance and provide
feedback after the presentation. The feedback must be constructive. It
should be directed to clarify the proposal's main concepts and propose
ways to improve the design.

\hypertarget{participate-actively-in-class.}{%
\subsubsection{6. Participate actively in
class.}\label{participate-actively-in-class.}}

Active engagement involves intervening during the discussion with
original ideas and open questions and linking the content of the
readings to the debate. Students are also encouraged to bring debates
that exceed the content of the session, but it is linked to the general
topic of the course.

\hypertarget{grading-policy}{%
\subsection{Grading policy}\label{grading-policy}}

\begin{itemize}
\item
  40\% of the grade is determined by the response papers.
\item
  40\% of the grade is determined by the presentation.
\item
  20\% of the grade is jointly determined by the quality and quantity of
  the participation in class and the discussion of the presentation.
\end{itemize}

\hypertarget{seminar-paper-6-credits}{%
\subsection{Seminar paper (6 credits)}\label{seminar-paper-6-credits}}

Students can choose to write a seminar paper to obtain six extra
credits. The seminar paper must have between 5500 and 6500 words.
Students can choose if they want to write an empirical or a theoretical
paper. The topic should be agreed upon between the instructor and the
student. To this end, students are asked to write a paper outline of 1-2
pages consisting of the following elements:

\begin{itemize}
\item
  Introduction of the topic
\item
  Research question
\item
  Academic and societal relevance
\item
  Theories (and hypotheses, if applicable)
\item
  Approach and structure of the paper (including a tentative empirical
  design, if applicable)
\end{itemize}

The deadline for the submission of the paper outline is \textbf{May 1st
2023}.

The deadline for the submission of the seminar paper is
\textbf{September 1st 2023}.

Please refer to ``The (Pro-)Seminar Paper'' section of the guidelines
for
\href{https://www.unilu.ch/fileadmin/fakultaeten/ksf/institute/polsem/Dok/Studium/2016-Jan_Guidelines_Booklet_engl.pdf}{Academic
Research and Writing of the Department of Political Science.} for more
details.

\hypertarget{office-hours}{%
\section{Office hours}\label{office-hours}}

Student meetings can be held during the instructor's \textbf{weekly
office hours}, scheduled each \textbf{Tuesday} (starting from February
21st until June 6th inclusive) *\textbf{between 13.30 and 15.30}. If any
student wants to make use of the office hours, she must book a time slot
via e-mail:
\href{mailto:alvaro.canalejo@unilu.ch}{\nolinkurl{alvaro.canalejo@unilu.ch}}.
The office is in \textbf{room 3.A12}.

Any student can send an e-mail during office hours without scheduling a
meeting for quick questions.

\hypertarget{guidelines-for-the-presentation}{%
\section{Guidelines for the
presentation}\label{guidelines-for-the-presentation}}

A good presentation should include the following:

\begin{enumerate}
\def\labelenumi{\arabic{enumi}.}
\item
  An outline of the presentation.
\item
  A brief summary of the readings, including their main argument,
  methodology (if applicable) and contribution. It should focus on those
  aspects of the text that the student(s) aim to analyze later. It
  should never be a full summary of the content.
\item
  A critique of the argument or the research design.

  \begin{itemize}
  \tightlist
  \item
    Some examples are an inconsistent logic, flawed conceptualizations
    (e.g., concepts that do not travel well across contexts), limited
    external validity (e.g., results that do not apply in other
    contexts), limited internal validity (e.g., poorly identified causal
    relationship, for example, due to omitted variables), limited
    relevance, etc.
  \end{itemize}
\item
  Propose a feasible research design to overcome these limitations or
  build upon them.

  \begin{itemize}
  \tightlist
  \item
    Matching the previous examples, some examples are a better-built
    argument that leads to different implications, an original
    conceptualization that performs better, outlining a research design
    that would prove the results valid in a different or wider setting,
    or a research design that controlled for potential confounders.
  \end{itemize}
\end{enumerate}

The presentation is evaluated based on its clarity, the understanding of
the original readings' concepts, the critical evaluation's depth, the
congruency between the critique and the research proposal's aim to solve
it, the argumentation and the ability to discuss the discussant's
points. For more guidelines, see the guidelines on
\href{https://www.unilu.ch/fileadmin/fakultaeten/ksf/institute/polsem/Dok/Studium/2016-Jan_Guidelines_Booklet_engl.pdf}{Academic
Research and Writing of the Department of Political Science.}

\hypertarget{course-schedule}{%
\section{Course schedule}\label{course-schedule}}

\hypertarget{introduction-to-the-course}{%
\subsection{Introduction to the
course}\label{introduction-to-the-course}}

\textbf{22.02.2023.} \textbf{\emph{Session 01. Introduction}}

No readings.

\hypertarget{block-i.-introduction-to-parties-and-party-systems}{%
\subsection{Block I. Introduction to parties and party
systems}\label{block-i.-introduction-to-parties-and-party-systems}}

\textbf{01.03.2023.} \textbf{\emph{Session 02. Mapping the terrain:
parties, electoral competition and party systems}}

\begin{itemize}
\tightlist
\item
  Mandatory readings:
\end{itemize}

Mair, Peter (1997). ``Party systems and structures of competition'\,'.
In:
\emph{Party system change: approaches and interpretations. Oxford University Press}.

\begin{itemize}
\tightlist
\item
  Suggested readings:
\end{itemize}

Mair, Peter and Cas Mudde (1998). ``The party family and its study'\,'.
In: \emph{Annual Review of Political Science} 1.1, pp.~211--229.

Sartori, Giovanni (1976).
\emph{Parties and party systems: A framework for analysis}. ECPR press.

\textbf{08.03.2023.} \textbf{\emph{Session 03. The formation of party
systems I: the institutional approach}}

\begin{itemize}
\tightlist
\item
  Mandatory readings:
\end{itemize}

Benoit, Kenneth (2006). ``Duverger's law and the study of electoral
systems'\,'. In: \emph{French Politics} 4, pp.~69--83.

Heath, Oliver and Adam Ziegfeld (2022). ``Why So Little Strategic Voting
in India?'\,' In: \emph{American Political Science Review} 116.4,
pp.~1523--1529.

\begin{itemize}
\tightlist
\item
  Suggested readings:
\end{itemize}

Riker, William H (1982). ``The two--party system and Duverger's law: An
essay on the history of political science'\,'. In:
\emph{American political science review} 76.4, pp.~753--766.

\textbf{15.03.2023.} \textbf{\emph{Session 04. The formation of party
systems II: the sociological approach}}

\begin{itemize}
\tightlist
\item
  Mandatory readings:
\end{itemize}

Lipset, Seymour Martin and Stein Rokkan (1967). ``Cleavage structures,
party systems, and voter alignments: an introduction'\,'. In:
\emph{Party systems and voter alignments: Cross--national perspectives}.
Free Press.

\begin{itemize}
\tightlist
\item
  Suggested readings:
\end{itemize}

Boix, Carles (2007). ``The emergence of parties and party systems'\,'.
In: \emph{The Oxford handbook of comparative politics}.

\hypertarget{block-ii.-party-system-change-in-we}{%
\subsection{Block II. Party system change in
WE}\label{block-ii.-party-system-change-in-we}}

\textbf{22.03.2023.} \textbf{\emph{Session 05. Party system change in WE
I: Green parties and the silent revolution}}

\emph{Exceptionally, this session will take place the 22nd of March from
13:30 to 15:15 in the room 3.A05.}

\begin{itemize}
\tightlist
\item
  Mandatory readings:
\end{itemize}

Inglehart, Ronald (1971). ``The silent revolution: Changing values and
political styles in advanced industrial society'\,'. In:
\emph{American Political Science Review} 65.4, pp.~991-1017.

Müller--Rommel, Ferdinand (1998). ``Explaining the electoral success of
green parties: A cross--national analysis'\,'. In:
\emph{Environmental Politics} 7.4, pp.~145--154.

\begin{itemize}
\tightlist
\item
  Suggested readings:
\end{itemize}

Müller--Rommel, Ferdinand (2019).
\emph{New politics in Western Europe: The rise and success of green parties and alternative lists}.
Routledge.

\textbf{22.03.2023.} \textbf{\emph{Session 06. Party system change in WE
II: New far right parties and the silent counter-revolution}}

\emph{Exceptionally, this session will take place the 22nd of March from
15:15 to 17:00 in the room 3.A05.}

\begin{itemize}
\tightlist
\item
  Mandatory readings:
\end{itemize}

Ignazi, Piero (1992). ``The silent counter--revolution'\,'. In:
\emph{European Journal of Political Research} 22.1, pp.~3--34.

Mudde, Cas (2014). ``Fighting the system? Populist radical right parties
and party system change'\,'. In: \emph{Party Politics} 20.2,
pp.~217--226.

\begin{itemize}
\tightlist
\item
  Suggested readings:
\end{itemize}

Betz, Hans--George (1993). ``The new politics of resentment: radical
right--wing populist parties in Western Europe'\,'. In:
\emph{Comparative politics}, pp.~413--427.

\textbf{05.04.2023.} \textbf{\emph{Session 07. Party system change III:
Economic voting after the Great Recession}}

\begin{itemize}
\tightlist
\item
  Mandatory readings:
\end{itemize}

Fiorina, Morris P (1978). ``Economic retrospective voting in American
national elections: A micro--analysis'\,'. In:
\emph{American Journal of Political Science}, pp.~426--443.

Kriesi, Hanspeter and Swen Hutter (2019). ``Economic and political
crises--the context of critical elections'\,'. In:
\emph{European Party Politics in Times of Crisis} , p.~33.

\begin{itemize}
\tightlist
\item
  Suggested readings:
\end{itemize}

Hutter, Swen, Hanspeter Kriesi, and Guillem Vidal (2018). ``Old versus
new politics: The political spaces in Southern Europe in times of
crises'\,'. In: \emph{Party politics} 24.1, pp.~10--22.

\hypertarget{block-iii.-demand-side-and-supply-side-explanations}{%
\subsection{Block III. Demand-side and supply-side
explanations}\label{block-iii.-demand-side-and-supply-side-explanations}}

\textbf{19.04.2023.} \textbf{\emph{Session 08. Demand-side changes I:
the emergence of a cultural cleavage}}

\begin{itemize}
\tightlist
\item
  Mandatory readings:
\end{itemize}

Kriesi, Hanspeter, Edgar Grande, Romain Lachat, Martin Dolezal, Simon
Bornschier, and Timotheos Frey (2006). ``Globalization and the
transformation of the national political space: Six European countries
compared'\,'. In: \emph{European Journal of Political Research} 45.6,
pp.~921--956.

Zollinger, Delia (2022). ``Cleavage Identities in Voters' Own Words:
Harnessing Open-Ended Survey Responses'\,'. In:
\emph{American Journal of Political Science}, pp.~1--48.

\begin{itemize}
\tightlist
\item
  Suggested readings:
\end{itemize}

Bornschier, Simon (2010). ``The new cultural divide and the
two--dimensional political space in Western Europe'\,'. In:
\emph{West European Politics} 33.3, pp.~419--444.

Hooghe, Liesbet and Gary Marks (2018). ``Cleavage theory meets Europe's
crises: Lipset, Rokkan, and the transnational cleavage'\,'. In:
\emph{Journal of European Public Policy} 25.1, pp.~109--135.

\textbf{26.04.2023.} \textbf{\emph{Session 09. Demand-side changes II:
old cleavages in the XXIst century}}

\begin{itemize}
\tightlist
\item
  Mandatory readings:
\end{itemize}

Goldberg, Andreas C (2020). ``The evolution of cleavage voting in four
Western countries: Structural, behavioural or political dealignment?'\,'
In: \emph{European Journal of Political Research} 59.1, pp.~68--90.

Oesch, Daniel (2008). ``Explaining workers' support for right--wing
populist parties in Western Europe: Evidence from Austria, Belgium,
France, Norway, and Switzerland'\,'. In:
\emph{International Political Science Review} 29.3, pp.~349--373.

\begin{itemize}
\tightlist
\item
  Suggested readings:
\end{itemize}

Oesch, Daniel and Line Rennwald (2018). ``Electoral competition in
Europe's new tripolar political space: Class voting for the left,
centre--right and radical right'\,'. In:
\emph{European journal of political research} 57.4, pp.~783--807. ISSN:
0304-4130.

\textbf{03.05.2023.} \textbf{\emph{Session 10. Supply-side changes I:
the weakening of traditional party ties}}

\begin{itemize}
\tightlist
\item
  Mandatory readings:
\end{itemize}

Mair, Peter (2013).
\emph{Ruling the void: The hollowing of Western democracy}. Verso Books.

Only chapters 1 and 3 are mandatory.

\begin{itemize}
\tightlist
\item
  Suggested readings:
\end{itemize}

Dalton, Russell J and Martin P Wattenberg (2002).
\emph{Parties without partisans: Political change in advanced industrial democracies}.
Oxford University Press on Demand.

Spoon, Jae--Jae and Heike Klüver (2019). ``Party convergence and vote
switching: Explaining mainstream party decline across Europe'\,'. In:
\emph{European Journal of Political Research}.

\textbf{10.05.2023.} \textbf{\emph{Session 11. Supply-side changes II:
challenger parties and the issue entrepreneurship model}}

\begin{itemize}
\tightlist
\item
  Mandatory readings:
\end{itemize}

Carmines, Edward G and James A Stimson (1986). ``On the structure and
sequence of issue evolution'\,'. In:
\emph{American Political Science Review} 80.3, pp.~901--920.

Hobolt, Sara B and Catherine E De Vries (2012). ``When dimensions
collide: The electoral success of issue entrepreneurs'\,'. In:
\emph{European Union Politics} 13.2, pp.~246--268.

\begin{itemize}
\tightlist
\item
  Suggested readings:
\end{itemize}

Vries, Catherine E de and Sara Hobolt (2020).
\emph{Political entrepreneurs: the rise of challenger parties in Europe}.
Princeton University Press.

\hypertarget{block-iv.-implications-and-empirical-applications}{%
\subsection{Block IV. Implications and empirical
applications}\label{block-iv.-implications-and-empirical-applications}}

\textbf{17.05.2023.} \textbf{\emph{Session 12. Assessing party system
change: the `freezing hypothesis' debate}}

\begin{itemize}
\tightlist
\item
  Mandatory readings:
\end{itemize}

Chiaramonte, Alessandro and Vincenzo Emanuele (2017). ``Party system
volatility, regeneration and de--institutionalization in Western Europe
(1945--2015)'\,'. In: \emph{Party politics} 23.4, pp.~376--388.

Mair, Peter (1993). ``Myths of electoral change and the survival of
traditional parties: The 1992 Stein Rokkan Lecture'\,'. In:
\emph{European Journal of Political Research} 24.2, pp.~121--133.

\begin{itemize}
\tightlist
\item
  Suggested readings:
\end{itemize}

Bartolini, Stefano and Peter Mair (1990).
\emph{Identity and availability. The Stabilization of the European Electorates, 1885-1985}.
Cambridge University Press, Cambridge.

\textbf{24.05.2023.} \textbf{\emph{Session 13. The consequences of party
system change}}

\begin{itemize}
\tightlist
\item
  Mandatory readings:
\end{itemize}

Bischof, Daniel and Markus Wagner (2019). ``Do Voters Polarize When
Radical Parties Enter Parliament?'\,' In:
\emph{American Journal of Political Science}.

Valentim, Vicente and Elias Dinas (2023). ``Does Party--System
Fragmentation Affect the Quality of Democracy?'\,' In:
\emph{British Journal of Political Science}.

\begin{itemize}
\tightlist
\item
  Suggested readings:
\end{itemize}

Canalejo-Molero, Álvaro (2023).
\emph{Disruptive Elections and their Implications for Satisfaction with Democracy}.

\textbf{31.05.2023.} \textbf{\emph{Session 14. Empirical workshop on
electoral volatility}}

No readings.

\end{document}
